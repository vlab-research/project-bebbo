\documentclass[a4paper,12pt]{article}
\usepackage[backend=biber, citestyle=authoryear, bibencoding=utf8]{biblatex}
\usepackage{amsmath, amsthm, amsfonts, mathtools, csquotes, bm, centernot, bbm, multirow}
\usepackage[toc,page]{appendix}
\usepackage[figuresright]{rotating}
\usepackage{geometry}
\geometry{a4paper, margin=2.5cm}

\theoremstyle{proposition}
\newtheorem{proposition}{Proposition}[section]
\newtheorem{prop}{Proposition}

\usepackage{pgf, tikz}
\usetikzlibrary{arrows, automata}

\DeclareMathOperator*{\argmax}{argmax}
\DeclareMathOperator*{\argmin}{argmin}

\title{The Impact of Bebbo: A Randomized Control Trial}

\begin{document}
% \maketitle

% \section{Balance}

% With N observations, N control and N treated, we look at balance across a set of observable characteristics of interest at baseline:

% 
% Table created by stargazer v.5.2.3 by Marek Hlavac, Social Policy Institute. E-mail: marek.hlavac at gmail.com
% Date and time: Sat, Jul 01, 2023 - 04:02:04 PM
\begin{table}[!htbp] \centering 
  \caption{} 
  \label{tbl:balance} 
\begin{tabular}{@{\extracolsep{5pt}} cccc} 
\\[-1.8ex]\hline 
\hline \\[-1.8ex] 
 & control & treatment & difference \\ 
\hline \\[-1.8ex] 
gender\_Man & $0.19$ & $0.13$ & $$-$0.06$ \\ 
gender\_Prefer not to answer & $0.02$ & $0.03$ & $0.01$ \\ 
gender\_Woman & $0.79$ & $0.84$ & $0.05$ \\ 
breastfed & $0.43$ & $0.44$ & $0.02$ \\ 
breastfed:\textless NA\textgreater  & $0.60$ & $0.58$ & $$-$0.02$ \\ 
past\_24h\_read & $0.80$ & $0.78$ & $$-$0.02$ \\ 
\hline \\[-1.8ex] 
\end{tabular} 
\end{table} 


% \clearpage
% \section{Results}


% \subsection{Difference in Differences Model}


% A simple difference-in-differences OLS regression on binary outcomes, no controls:


% Table created by stargazer v.5.2.3 by Marek Hlavac, Social Policy Institute. E-mail: marek.hlavac at gmail.com
% Date and time: Tue, Jul 11, 2023 - 02:54:34 AM
% Requires LaTeX packages: rotating 
\begin{sidewaystable}[!htbp] \centering 
  \caption{} 
  \label{tbl:regression} 
\begin{tabular}{@{\extracolsep{5pt}}lccccc} 
\\[-1.8ex]\hline 
\hline \\[-1.8ex] 
 & \multicolumn{5}{c}{\textit{Dependent variable:}} \\ 
\cline{2-6} 
\\[-1.8ex] & dev\_knw\_recog & dev\_knw\_concern\_0\_2 & health\_knw & practices\_24 & was\_breastfed \\ 
\\[-1.8ex] & (1) & (2) & (3) & (4) & (5)\\ 
\hline \\[-1.8ex] 
 treatmenttreated & 0.02 & 0.02 & 0.002 & $-$0.01 & $-$0.01 \\ 
  & (0.02) & (0.02) & (0.04) & (0.02) & (0.07) \\ 
  & & & & & \\ 
 endline & 0.06$^{***}$ & $-$0.01 & 0.04 & 0.02 & $-$0.001 \\ 
  & (0.02) & (0.02) & (0.04) & (0.01) & (0.07) \\ 
  & & & & & \\ 
 treatmenttreated:endline & $-$0.001 & $-$0.002 & 0.02 & 0.01 & $-$0.01 \\ 
  & (0.03) & (0.03) & (0.05) & (0.02) & (0.09) \\ 
  & & & & & \\ 
 Constant & 0.88$^{***}$ & 0.65$^{***}$ & 0.84$^{***}$ & 0.68$^{***}$ & 0.43$^{***}$ \\ 
  & (0.01) & (0.01) & (0.03) & (0.01) & (0.05) \\ 
  & & & & & \\ 
\hline \\[-1.8ex] 
Observations & 1,102 & 449 & 449 & 1,101 & 449 \\ 
\hline 
\hline \\[-1.8ex] 
\textit{Note:}  & \multicolumn{5}{r}{$^{*}$p$<$0.1; $^{**}$p$<$0.05; $^{***}$p$<$0.01} \\ 
\end{tabular} 
\end{sidewaystable} 


% \subsection{Instrumental Variable Model}

% An instrumental variable regression using randomization as the instrument. Some notes:

% \begin{enumerate}
% \item We assume monotonicity in that asking parents to download and use Bebbo should not prevent a parent from using Bebbo that otherwise would.
% \item We rely on a follow-up variable to measure Bebbo usage in the control group (spillovers) due to the fact that the app does exist in the country and anyone could have it and use it.
% \item The first stage measures the impact of randomization on Bebbo usage.
% \item The second stage regresses that impact on the outcome of interest.
% \end{enumerate}




\end{document}