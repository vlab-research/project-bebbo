\documentclass{article}
\usepackage{adjustbox}
\usepackage{graphicx}
\usepackage{geometry}
\usepackage{dcolumn}
 \geometry{
 a4paper,
 total={170mm,257mm},
 left=10mm,
 top=10mm
 }

\title{Testing for Baseline Imbalance}
\date{September 2023}

\begin{document}

\section*{Testing for Baseline Imbalance by Country}
\subsubsection*{Methodology}
\begin{itemize}
    \item We test for differences in the means between countries of all the survey question responses, construct variables and demographic variables.
    \item We filter for baseline respondents who are also present in endline.
    \item We run independent t-tests one at a time for each variable being tested.
    \item We assume that the samples from both countries are being drawn from populations of equal variance.
    \item $N_{serbia} = 817$ and $N_{bulgaria} = 284$
\end{itemize}




% Table created by stargazer v.5.2.3 by Marek Hlavac, Social Policy Institute. E-mail: marek.hlavac at gmail.com
% Date and time: Thu, Sep 21, 2023 - 10:56:19
\begin{table}[!htbp] \centering 
  \caption{Baseline imbalance by country} 
  \label{tbl:country_baseline_constructs_imbalance} 
  \resizebox{\columnwidth}{!}{%
\begin{tabular}{@{\extracolsep{3pt}} ccccccccc} 
\\[-1.8ex]\hline 
\hline \\[-1.8ex] 
variable & serbia mean & bulgaria mean & test statistic & p-val & std.error & conf.int lower & conf.int upper & df \\ 
\hline \\[-1.8ex] 
health\_knw & 0.81 & 0.739 & 4.669 & 0 & 0.015 & 0.041 & 0.101 & 2074 \\ 
dev\_knw\_recog & 0.864 & 0.823 & 4.849 & 0 & 0.009 & 0.025 & 0.058 & 4855 \\ 
confidence & 3.392 & 3.256 & 7.056 & 0 & 0.019 & 0.098 & 0.174 & 4777 \\ 
attitude & 3.067 & 3.231 & -6.61 & 0 & 0.025 & -0.212 & -0.115 & 4739 \\ 
caregiver\_well\_being & 2.905 & 3.002 & -5.809 & 0 & 0.017 & -0.13 & -0.064 & 4724 \\ 
was\_breastfed & 0.405 & 0.298 & 4.801 & 0 & 0.022 & 0.064 & 0.152 & 1909 \\ 
practices\_24 & 0.839 & 0.784 & 8.809 & 0 & 0.006 & 0.043 & 0.068 & 4507 \\ 
practices\_agree & 2.94 & 1.421 & 71.916 & 0 & 0.021 & 1.478 & 1.56 & 4483 \\ 
practices\_hostility & 3.058 & 1.98 & 51.99 & 0 & 0.021 & 1.037 & 1.118 & 4465 \\ 
parent\_age & 31.704 & 31.039 & 3.055 & 0.002 & 0.218 & 0.238 & 1.092 & 8579 \\ 
 &  &  &  &  &  &  &  &  \\ 
\hline \\[-1.8ex] 
\end{tabular} %
}
\end{table} 



\section*{Testing for Baseline Imbalance by Treatment}
\subsubsection*{Methodology}

\begin{itemize}
    \item We test for differences in the means between treatment and control of all the survey question responses, construct variables and demographic variables.
    \item We filter for baseline respondents who are also present in endline.
    \item We run independent t-tests one at a time for each variable being tested.
    \item We assume that the samples from both countries are being drawn from populations of equal variance.
    \item We repeat this analysis for both countries, Serbia and Bulgaria
\end{itemize}



% Table created by stargazer v.5.2.3 by Marek Hlavac, Social Policy Institute. E-mail: marek.hlavac at gmail.com
% Date and time: Mon, Sep 11, 2023 - 08:46:02
\begin{table}[!htbp] \centering 
  \caption{} 
  \label{tbl:treatment_baseline_imbalance_serbia} 
\begin{tabular}{@{\extracolsep{5pt}} cccccc} 
\\[-1.8ex]\hline 
\hline \\[-1.8ex] 
variable & test statistic & p-val & std.error & conf.int lower & conf.int upper \\ 
\hline \\[-1.8ex] 
multiplication & 2.769 & 0.006 & 0.024 & 0.02 & 0.115 \\ 
concern\_3yo & 2.412 & 0.016 & 0.029 & 0.013 & 0.127 \\ 
sort\_pair & -1.938 & 0.054 & 0.053 & -0.208 & 0.002 \\ 
know\_cog\_dev & 1.88 & 0.06 & 0.022 & -0.002 & 0.086 \\ 
dev\_knw\_concern\_3\_6 & 1.871 & 0.062 & 0.021 & -0.002 & 0.08 \\ 
simple\_songs & 1.67 & 0.096 & 0.029 & -0.008 & 0.104 \\ 
parent\_age & 1.545 & 0.123 & 0.491 & -0.205 & 1.724 \\ 
dev\_knw\_recog & 1.54 & 0.124 & 0.017 & -0.007 & 0.06 \\ 
concern\_5yo & 1.494 & 0.136 & 0.042 & -0.02 & 0.146 \\ 
scribble & 1.353 & 0.177 & 0.042 & -0.026 & 0.141 \\ 
sit\_support & 1.343 & 0.18 & 0.053 & -0.033 & 0.176 \\ 
know\_social\_emotional\_dev & 1.32 & 0.187 & 0.022 & -0.014 & 0.074 \\ 
past\_24h\_read & -1.227 & 0.22 & 0.027 & -0.087 & 0.02 \\ 
lose\_patience\_punish & -1.219 & 0.223 & 0.068 & -0.215 & 0.05 \\ 
express\_feelings & 1.104 & 0.271 & 0.045 & -0.039 & 0.139 \\ 
know\_phys\_dev & 1.055 & 0.292 & 0.02 & -0.018 & 0.061 \\ 
snap\_at\_child & 0.992 & 0.321 & 0.063 & -0.061 & 0.187 \\ 
past\_24h\_draw & -0.969 & 0.333 & 0.033 & -0.095 & 0.032 \\ 
know\_which\_vaccine & -0.86 & 0.39 & 0.047 & -0.133 & 0.052 \\ 
tell\_story & 0.855 & 0.393 & 0.028 & -0.031 & 0.078 \\ 
practices\_24 & -0.82 & 0.413 & 0.014 & -0.038 & 0.016 \\ 
concern\_12mo & -0.764 & 0.445 & 0.044 & -0.12 & 0.053 \\ 
health\_knw & -0.705 & 0.481 & 0.034 & -0.092 & 0.043 \\ 
make\_fun\_of & -0.684 & 0.494 & 0.066 & -0.176 & 0.085 \\ 
say\_name\_age & -0.632 & 0.528 & 0.049 & -0.128 & 0.066 \\ 
know\_lang\_dev & 0.625 & 0.532 & 0.021 & -0.028 & 0.055 \\ 
count\_to\_ten & 0.6 & 0.549 & 0.041 & -0.056 & 0.105 \\ 
parenting\_stress\_2 & 0.579 & 0.563 & 0.044 & -0.061 & 0.112 \\ 
cry\_separated & -0.52 & 0.604 & 0.04 & -0.1 & 0.058 \\ 
movement & 0.494 & 0.621 & 0.033 & -0.048 & 0.081 \\ 
past\_24h\_sing & 0.468 & 0.64 & 0.022 & -0.033 & 0.054 \\ 
past\_24h\_outside & -0.46 & 0.646 & 0.025 & -0.06 & 0.037 \\ 
know\_name\_age & 0.436 & 0.663 & 0.023 & -0.036 & 0.056 \\ 
name\_colors & 0.432 & 0.666 & 0.014 & -0.022 & 0.034 \\ 
practices\_hostility & -0.424 & 0.672 & 0.046 & -0.109 & 0.07 \\ 
breastfed & 0.251 & 0.802 & 0.054 & -0.093 & 0.12 \\ 
was\_breastfed & 0.251 & 0.802 & 0.054 & -0.093 & 0.12 \\ 
decrease\_stress & -0.235 & 0.815 & 0.029 & -0.063 & 0.049 \\ 
self\_care & -0.235 & 0.815 & 0.029 & -0.063 & 0.049 \\ 
know\_when\_vaccine & -0.232 & 0.817 & 0.035 & -0.076 & 0.06 \\ 
caregiver\_well\_being & 0.221 & 0.825 & 0.037 & -0.065 & 0.082 \\ 
confidence\_respond\_misbehave & -0.217 & 0.828 & 0.053 & -0.115 & 0.092 \\ 
threaten & -0.203 & 0.84 & 0.062 & -0.133 & 0.108 \\ 
joke\_with\_child & -0.194 & 0.846 & 0.071 & -0.154 & 0.126 \\ 
alphabet & 0.189 & 0.85 & 0.044 & -0.078 & 0.094 \\ 
physical\_punishment & 0.184 & 0.854 & 0.053 & -0.094 & 0.113 \\ 
attitude & 0.184 & 0.854 & 0.053 & -0.094 & 0.113 \\ 
parent\_knw & -0.181 & 0.856 & 0.046 & -0.099 & 0.082 \\ 
past\_24h\_play & 0.126 & 0.9 & 0.015 & -0.028 & 0.032 \\ 
past\_24h\_stories & -0.119 & 0.906 & 0.023 & -0.048 & 0.043 \\ 
smile\_around\_child & 0.114 & 0.909 & 0.07 & -0.129 & 0.145 \\ 
confidence\_deal\_emotions & -0.105 & 0.917 & 0.05 & -0.104 & 0.093 \\ 
parenting\_stress\_1 & 0.089 & 0.929 & 0.053 & -0.1 & 0.109 \\ 
personal\_needs & -0.082 & 0.935 & 0.057 & -0.116 & 0.106 \\ 
meaning\_of\_no & -0.081 & 0.935 & 0.044 & -0.09 & 0.083 \\ 
improve\_family & 0.075 & 0.94 & 0.032 & -0.06 & 0.064 \\ 
family\_care & 0.075 & 0.94 & 0.032 & -0.06 & 0.064 \\ 
play\_on\_floor & 0.041 & 0.968 & 0.071 & -0.136 & 0.141 \\ 
laugh\_together & 0.039 & 0.969 & 0.075 & -0.144 & 0.15 \\ 
dev\_knw\_concern\_0\_2 & 0.021 & 0.983 & 0.015 & -0.028 & 0.029 \\ 
practices\_agree & 0 & 1 & 0.055 & -0.108 & 0.108 \\ 
\hline \\[-1.8ex] 
\end{tabular} 
\end{table} 



% Table created by stargazer v.5.2.3 by Marek Hlavac, Social Policy Institute. E-mail: marek.hlavac at gmail.com
% Date and time: Thu, Sep 21, 2023 - 10:56:02
\begin{table}[!htbp] \centering 
  \caption{} 
  \label{tbl:treatment_baseline_imbalance_bulgaria} 
\begin{tabular}{@{\extracolsep{5pt}} ccccccccc} 
\\[-1.8ex]\hline 
\hline \\[-1.8ex] 
variable & treatment mean & control mean & test statistic & p-val & std.error & conf.int lower & conf.int upper & df \\ 
\hline \\[-1.8ex] 
dev\_knw\_recog & 0.841 & 0.804 & 2.579 & 0.01 & 0.014 & 0.009 & 0.064 & 1863 \\ 
parent\_age & 31.333 & 30.721 & 1.685 & 0.092 & 0.363 & -0.1 & 1.325 & 3754 \\ 
practices\_24 & 0.791 & 0.776 & 1.433 & 0.152 & 0.011 & -0.006 & 0.036 & 1689 \\ 
attitude & 3.258 & 3.202 & 1.409 & 0.159 & 0.04 & -0.022 & 0.134 & 1804 \\ 
practices\_hostility & 1.999 & 1.961 & 1.158 & 0.247 & 0.033 & -0.026 & 0.102 & 1676 \\ 
confidence & 3.269 & 3.243 & 0.864 & 0.388 & 0.03 & -0.033 & 0.085 & 1826 \\ 
caregiver\_well\_being & 3.012 & 2.991 & 0.786 & 0.432 & 0.027 & -0.031 & 0.074 & 1797 \\ 
health\_knw & 0.748 & 0.729 & 0.772 & 0.441 & 0.025 & -0.03 & 0.068 & 829 \\ 
was\_breastfed & 0.286 & 0.31 & -0.739 & 0.46 & 0.033 & -0.09 & 0.041 & 747 \\ 
practices\_agree & 1.426 & 1.416 & 0.515 & 0.607 & 0.02 & -0.028 & 0.048 & 1681 \\ 
 &  &  &  &  &  &  &  &  \\ 
\hline \\[-1.8ex] 
\end{tabular} 
\end{table} 



\end{document}